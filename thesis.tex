%%
%% This is file `thesis.tex',
%% generated with the docstrip utility.
%%
%% The original source files were:
%%
%% hqugxythesis.dtx  (with options: `thesis')
%% 
%% This is a generated file.
%% 
%% This file may be distributed and/or modified under the
%% conditions of the LaTeX Project Public License, either version 1.3a
%% of this license or (at your option) any later version.
%% The latest version of this license is in:
%% 
%% http://www.latex-project.org/lppl.txt
%% 
%% and version 1.3a or later is part of all distributions of LaTeX
%% version 2004/10/01 or later.
%% 
%% To produce the documentation run the original source files ending with `.dtx'
%% through LaTeX.
%% 




\documentclass{hqugxythesis}
\usepackage{lipsum}
\usepackage{zhlipsum}
\title{论文标题}
\author{我是谁}
\renewcommand\etitle{English Title} % 英文标题
\renewcommand\eauthor{Whoami} % 英文作者
\renewcommand\studentid{123456789} % 学号
\renewcommand\graduateyear{2022} %毕业年份
\renewcommand\major{信息工程专业}
\renewcommand\emajor{Electrical Engineering}


\addbibresource{refs.bib}
\pagestyle{plain}


\begin{document}
\pagenumbering{Roman}
\xiaosi[1.5]\selectfont


{
  \setlength{\parindent}{0em} % 重设段落缩进
  \mkchtitle
  \begin{cabstract}
    \zhlipsum[1-1]
  \end{cabstract}
  \ckeywords{物联网; 目标识别; clojure; }

  \newpage

  \mkentitle
  \begin{eabstract}
    \lipsum[1-1]
  \end{eabstract}
  \ekeywords{IoT; Object Detection; clojure; }
}

\tableofcontents


\chapter{绪论}
\pagenumbering{arabic}
\setcounter{page}{1}
\xiaosi[1.5]\selectfont
\section{本课题的的研究目的和意义}
引文\cite{canny_john}
\section{国内外研究现状}
\zhlipsum
\chapter{总结与展望}
\zhlipsum


\addcontentsline{toc}{chapter}{参考文献}
\printbibliography[heading=bibliography, title=参考文献]


\begin{ack}
  \zhlipsum
\end{ack}

\end{document}



